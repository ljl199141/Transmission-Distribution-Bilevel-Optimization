%%!TEX root = Journal.tex
%The power grid in the 21st Century is undergoing great structural and operational modernization. The grid is transitioning from a centralized generation and control structure towards a decentralized one as more and more distributed energy resources and renewable generation emerge. Consequently, MGs, defined by the Department of Energy as 'localized grids that can disconnect from the traditional grid to operate autonomously', have been emerging in recent years. The autonomous operation of the MGs adds great resilience and reliability to the nation's electrical power grid for dealing with disturbances or natural disasters\cite{MG}. In addition, MGs could also alleviate transmission congestion\cite{wsc},  reduce transmission and  distribution losses, and provide DR(DR)\cite{MG}.  Those characteristics make MGs a basic building block of a decentralized and robust future grid. The future distribution network could comprise many MGs, each serving a small part of the distribution network. Accordingly, the distribution network will operate differently in many ways.


%MGs, defined by the U.S. Department of Energy as `localized grids that can disconnect from the traditional grid to operate autonomously', have been emerging in recent years. The autonomous operation of MGs adds  resilience and reliability to the nation's electrical power grid for dealing with disturbances or natural disasters\cite{MG}. In addition, MGs could also alleviate transmission congestion\cite{wsc},  reduce transmission losses, and provide DR (DR)\cite{MG}. These characteristics make MGs a basic building block of a decentralized and robust future grid. The future distribution network could be comprised of many MGs, which would require the networks to operate differently in various ways. A key difference will be the implementation of DR. Currently, there are aggregators providing  DR to customers in the distribution network. In this paradigm, the role of DR aggregator could be assumed by MGs with the implementation of various programs for its customers. 

%In this paper, three types of DR are explored for a MG. They are based on specific load classes and including thermostatically controlled load (TCL), deferrable load (DL) and elastic load (EL). These DR types are implemented and compared using a stochastic rolling horizon framework to optimize the operation of the MG in the real time market.
% including renewable generation and storage units.
Future power systems is envisioned to move towards a sustainable and decentralized system with a high penetration of renewable energy and distributed generation \cite{anderson2017vision}, MGs will play a critical role in this evolution as they feature distributed local generation, energy storage, and demand response and cater the need of increased need for flexibility of the future systems. With the possibility of numerous MG connected to the TS and exchanging power with the TS, the operations of power systems will change dramatically. Each MG could be view as an independent entity with its own operation objective which could conflict with the TS's. A proper way to co-optimize the two systems will thus become very important for the efficient operation of the whole system in the future.

The co-optimization of TS and MG operation is a relatively new area. What has been done falls into two categories. One category \cite{optimals,ref5,a,model,optimizing,population,risk,RTP2} focuses on the optimization of the MG operation with an abstract model for the TS. Specifically, this category models the TS as a node which connects to the MG and could buy or sell certain amount of energy from or to the MG at fixed prices. Non-consideration of the TS characteristics makes such models less realistic as the nodal energy exchange and price information is based on the detailed model and can change over time. As a result, those models make the optimization of the MG less practical and are unable to reflect the MG's impact on the TS. The other category \cite{CBA, CF, PSO,graph,lmprank, sto, sa, solar} focuses on the TS details with an over-simplified model for the MG. In those work, MGs are normally modeled as distributed generations (DG) which can only sell energy. In reality, a MG model can not only sell energy, but also buy energy from the TS, store energy, and vary its load consumption (ie: providing DR). Therefore, MGs often have their own energy management systems and their operation should ideally be modeled as a separate optimization problem. Treating MGs as DGs once again fails to capture the true impact of MGs on the TS and does not provide any insights about the influence of TS over the MG's operation. In the future where there will very likely be many MGs connected to the TS, the drawbacks of the above two modeling approaches will make them inapplicable. In addition, the impact of DR on the TS can not be effectively studied with those two approaches as DR happens at the distribution level and requires a detailed model of the distribution side. 

In this work, the operation of the TS and MG are optimized at the same time with a detailed model for each. A bilevel optimization approach is applied for this framework. Bilevel optimization is defined as a mathematical optimization problem in which one optimization problem (upper level problem) contains another optimization problem (lower level problem) \cite{bilevel1}. There are two types of approaches to solve bilevel optimization problems. The first category applies classical methods including single-level reduction \cite{bialas1984two,bard1982explicit}, descent methods \cite{kolstad1990derivative,savard1994steepest}, penalty function methods \cite{lv2007penalty,white1993penalty}, and trust-region methods \cite{colson2005trust,marcotte2001trust}. Most of the classical approaches deal with problems that are mathematically well-behaved; i.e., has convexity. Some strong assumptions such as continuous differentiability and lower semi-continuity are also quite common. Of those methods, the single level reduction method is the most widely used one when the lower level problem is convex and sufficiently regular. The single level reduction method is applied in this work. The second category applies evolutionary methods including genetic algorithms \cite{mathieu1994genetic}, particle swarm optimization \cite{li2006hierarchical}, differential evolution \cite{zhu2006hybrid}, and metamodeling-based methods \cite{wang2007review}. They normally require great computational efforts and do not provide performance guarantee. For a detailed review of different bilevel optimization methods, please refer to \cite{bilevel1, bilevel2}. 

%The TS model has a network with power flow constraints as well as renewable generation. To manage with the uncertainty of renewable generation, the generator reserves in the TS as well as the MG DR are used. The optimization of the reserve provision is also the optimization of the ancillary service market. One innovation of this work is that the DR is totally decoupled from the TS and is entirely modeled at the distribution system level MG. This is more realistic than the common DR modeling at the transmission level. 

In this work, the TS and the MG both are modeled in details. Bilevel optimization technique is used to co-optimize the two systems. In this optimization framework, the factors that affect the renewable (ie: wind in this study) penetration level in the TS and the operation cost of the two systems are analyzed. Recommendations on how to improve the wind penetration level and reduce the system operation cost are given. The key contributions of this work are summarized below:
\begin{enumerate}[(1)]
\item Bilevel optimization framework is proposed for the co-optimization of TS and MG as well as the co-optimization of the energy market and ancillary service market. 
\item A detailed TS model with network and detailed MG model are for the first time applied in the co-optimization framework.
\item DR is decoupled from the TS level and modeled at the distribution level, which provides a more realistic optimization model.
\item Factors that affect the system wind penetration level and operation cost are reviewed. 
%\item Recommendations are given to maximize the wind penetration and reduce the system operation cost. 
\end{enumerate}

The structure of the paper is as follows; the bilevel optimization model is discussed in section~\ref{sec:Model}.  Numerical results for  are reported in Section~\ref{sec:Results}. Concluding remarks and future research directions are given in Section~\ref{sec:Conclude}. 


%There has beenwithIn \cite{kun2015considering}, the author uses Karush?Kuhn?Tucker (KKT) conditions to solve the upper level problem of the TS and lower level problem of the MG. The influence of energy storage systems on the system cost is analyzed.  An islanded MG operation  problem with fixed load, transferable load, and user-action load is studied in \cite{ref5}. \cite{a} suggests a DR model that uses aggregated heat pumps and proposes a control algorithm that uses the DR and energy storage to provide tie-line smoothing service.  In \cite{model}, a genetic algorithm is applied to solve for the optimal distribution system operation and analyze the DR effect of MGs on AC load peak shaving. The DR model designed in \cite{optimizing} is based on load sharing between prosumers in a MG and is solved by a distributed iterative algorithm. In \cite{population}, a population game method is used to design the DR and achieve the optimal dispatch of power in a MG. A rolling horizon optimization framework for MG energy management with single renewable generation forecast is explored in \cite{mgrh} at the presence of shiftable load.
%\cite{risk} solves a two-stage stochastic optimization problem for a MG with voluntary and involuntary load curtailment to achieve the optimal bidding in the day-ahead market. DR based on load elasticity in a MG under real-time price is explored in \cite{RTP2}.
%In \cite{stochastic}, the author models DR as a type of generation or negative load that could provide reserve services via a two-stage stochastic MG optimization problem.
%Incorporating transferable load in an distribution system and using a DC MG to serve the DC load is shown to reduce peak demand and operation cost of the distribution system in \cite{utility}. 
%
%A comprehensive MG is considered in this work. The MG consists of a generator, a storage unit, aggregated loads of various types, and a wind farm, and is able to exchange power with the main grid. Since a MG typically covers a small local area with a power capacity of a few MWs \cite{the2}, and a system of this size has very limited power in the setting of the market price, it is assumed that the MG is a price taker in this work. Three types of DR are explored for a MG. They are based on specific load classes and including thermostatically controlled load (TCL), deferrable load (DL) and elastic load (EL). The work seeks to optimize the operation of the MG and interaction with the main network with the presence of those types of DRs to lower the MG operation cost.
% 
%Determining the optimal operation of a MG equipped with DR requires the consideration of specific characteristics of the MG. Since future MGs will likely incorporate renewable energy, which is highly unpredictable, a stochastic optimization approach is appropriate. As real time pricing is the driving force behind DR, and the real time market clears hour by hour, a rolling horizon approach with constantly updated renewable and pricing information is a natural choice for the real time MG optimal operation. None of the operation methods in the current literature address these characteristics jointly, which limits practicality for real MG use. 
%
%%Our work proposes to address those MG characteristics through a stochastic rolling horizon MG optimization structure. The main contributions are as follows:
%In contrast to previous work, this paper models the MG through a stochastic rolling horizon framework. This approach allows the following innovations: 
%
%1. A stochastic rolling horizon approach is proposed for the first time for the operation of a comprehensive MG with renewables, storage, and self-generation. This approach matches the characteristics of the real time market and the uncertain nature of renewable generation.
%
%2. Realistic DR classes are designed according to the classification of real load data to offer the most load flexibility.
%
%3. The performance of DR is analyzed under various historical scenarios. The analysis provides insights about the best use of the MG DR for the system operation.
%
%4. A real case study using three months of pricing, load and wind data from the ComEd's region is carried out to accurately represent different system conditions.
%
%The remainder of this paper is organized as follows. Section~\ref{sec:Demand} deals with the various types of loads, associated DR possibilities, and delineates the associated constraints for each type. These constraints will then be implemented in the stochastic rolling horizon optimization model, as described in Section~\ref{sec:Model}. Section~\ref{sec:Wind} discusses the scenario generation and reduction strategy used to represent uncertainty.  Numerical results for the ComEd case study are reported in Section~\ref{sec:Results}. Concluding remarks and future research directions are given in Section~\ref{sec:Conclude}. 
