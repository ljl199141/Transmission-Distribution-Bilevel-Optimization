%!TEX root = HICSS51.tex
\subsection{Big-M Method}
The complementary conditions are set of constraints with the following format:\\
\begin{align}
\lambda_i*g_i(x,y) =0 \nonumber
\end{align}
Specifically, given a sufficiently large positive value $M_i$ and binary value $\phi_i$, the complementary conditions could be reformulated as below:
\begin{align}
&-(1-\phi_i)*M_i\leq g_i(x,y)\nonumber\\
&\lambda_i\leq\phi_i*M_i\nonumber
\end{align}

For a detailed treatment of the Big-M method, please refer to \cite{winston2003introduction}.

\subsection{Traditional UC Problem}
The formulation of the traditional UC problem adopted by the ISOs is based on the UC formulation in \cite{papavasiliou2013multiarea}. The optimization variables are the sum of the optimization variables from the two levels of the bilevel problem without the energy import price $c^{im}_t$ and DR price $p^{dr}_t$. 

\begin{align*}
x_t=[&w_{g,t}, p_{g,t}, r_{g,t}^{up}, r_{g,t}^{dn}, p^{dr}_{t}, c^{im}_{t}]
\end{align*}

Similarly, the objective function is the sum of the upper level and lower level bilevel optimization objective function without the DR and energy import payment terms.

\textbf{\emph{Objective function:}}
\begin{align*}
F(\{x_t\}^{T}_{t=1}) =& \sum_{t=1}^{T}\sum_{g=1}^{G}(C^c_{g,t} w_{g,t}+C_{g,t} p_{g,t}+C^r_{g}(r_{g,t}^{up}+r_{g,t}^{dn}))\\
+ &\sum_{t=1}^{T}\sum_{i=1}^{N^d_b}(C^{d1}_i p^d_{i,t} +C^{d2}_g p^d_{i,t}  p^d_{i,t} \\
+ &C^{dr1} (dr^{up}_{i,t} + dr^{dn}_{i,t} ) \\
+ &C^{dr2} (dr^{up}_{i,t}dr^{up}_{i,t} + dr^{dn}_{i,t}dr^{dn}_{i,t}  ) \\
+ &C^{d}(\overline{P}^d_{i}-p^d_{i,t})(\overline{P}^d_{i}-p^d_{i,t}))
\end{align*}

The constraints of the traditional approach with DS network constraints are the same as the ones used in the bilevel optimization approach. The formulation is as below:

\begin{subequations}
\begin{align}
\text{min}_{\{x_t\}^{T}_{t=1}} & F\left(\{x_t\}^{T}_{t=1}\right)\nonumber\\
\text{s.t.}  & (2) - (15)\nonumber
\end{align}
\end{subequations}

If the DS power flow constraints namely Eqn. (10) \& (11) are taken out of the constraint set, the problem becomes the tradition UC without DS network constraints version.

%\subsection{Mccormick Envelope}
%For a toy optimization problem of the following format:\\
%\begin{align}
%\text{min } Z = x*y\\ \nonumber
%\underline{x} \leq x \leq \overline{x} \nonumber\\
%\underline{y} \leq y \leq \overline{y} \nonumber
%\end{align}
%The problem could be reformulated by introducing a new variable $w=x*y$ as follow:\\
%\begin{align}
%\text{min } Z = w\\ \nonumber
%w \geq \overline{x} *y +x*\overline{y} -\overline{x} *\overline{y} \nonumber\\  
%w \geq \underline{x} *y +x*\underline{y} -\underline{x} *\underline{y} \nonumber\\  
%w \leq \overline{x} *y +x*\underline{y} -\overline{x} *\underline{y} \nonumber\\
%w \leq \underline{x} *y +x*\overline{y} -\underline{x} *\overline{y} \nonumber\\
%\underline{x} \leq x \leq \overline{x} \nonumber\\
%\underline{y} \leq y \leq \overline{y} \nonumber
%\end{align}
%For a detailed treatment of the Big-M method, please refer to \cite{castro2015tightening}.

\subsection{Microgrid Parameters}
The parameter values for a 25 MW MG are listed in the table below.

\begin{table}[h]
\centering
\begin{tabular}{ |c|c|c|c|c|c|c| } 
 \hline
Parameters & Bus 1 & Bus 2& Bus 3& Bus 4& Bus 5& Bus 6\\ 
 \hline
$\overline{L}^d_{i}$ & 0MW & 2MW & 2.5MW & 0MW & 2MW & 1MW\\ 
 \hline
$\overline{L}^d_{i}$ & 0MW & 1MW & 1.25MW & 0MW & 1MW & 0.5MW  \\ 
 \hline
$L^{ie}_{i}$ & 0MW & 2MW & 2.5MW & 0MW & 2MW & 1MW  \\ 
 \hline
$\overline{\overline{P}_{i}}$ & 10MW & 10MW & 10MW & 10MW & 10MW & 10MW  \\ 
 \hline
$\underline{\underline{P}_{i}}$ & 0MW & 0MW & 0MW & 0MW & 0MW & 0MW  \\ 
 \hline
$C^{d1}_{i}$ & 0\$/MW & 0\$/MW & 0\$/MW & 0\$/MW & 0\$/MW & 5\$/MW   \\ 
 \hline
$C^{d2}_{i}$ & 0\$/MW & 0\$/MW & 0\$/MW & 0\$/MW & 0\$/MW & 0.02\$/MW   \\ 
 \hline
$\overline P^d_i$ & 0MW & 0MW & 0MW & 0MW & 0MW & 10MW \\ 
 \hline
 $\underline P^d_i$ & 0MW & 0MW & 0MW & 0MW & 0MW & 0MW  \\ 
 \hline
\end{tabular}

\begin{tabular}{ |c|c|c|c|c|c|c| } 
 \hline

\end{tabular}
 
\begin{tabular}{ |c|c|c|c|c|c|c| } 
 \hline
Parameters & Values\\ 
 \hline
$C^{d}_t$ & 3\$/MW  \\ 
 \hline
 $C^{dr1}_t$ & 1\$/MW  \\ 
 \hline
$C^{dr2}_t$ & 1\$/MW  \\ 
 \hline
\end{tabular}
\caption{DS parameter values}
 \label{wdlevel}
\end{table} 