%!TEX root = HICSS51.tex
With the advance in smart grid technologies, the electricity market has been transforming from a centralized market to a deregulated market, from a vertical structure to a horizontal structure due to some driving factors such as renewable generation, demand response and distributed generation \cite{nesamalar2016energy}. Consequently, the market is becoming more dynamic and competitive \cite{sun2017identifying}. To suit this deregulated market nature, it is paramount for the three main players in the market namely generation, transmission and distribution to optimize their operation strategies by modeling the interactions between themselves and the other market players. This interaction could be essentially viewed from a game-theory perspective and could be modeled through some game structure. Specifically, bilevel optimization which is closely related to the Stackelberg game in economics is widely applied to model the interaction among the three players \cite{colson2007overview}. 

Bilevel optimization is defined as a mathematical optimization problem in which one optimization problem (upper level problem) contains another optimization problem (lower level problem) \cite{bilevel1}. There are two types of approaches to solve bilevel optimization problems. The first category applies classical methods including single-level reduction \cite{bialas1984two,bard1982explicit}, descent \cite{kolstad1990derivative,savard1994steepest}, penalty function \cite{lv2007penalty,white1993penalty}, and trust-region methods \cite{colson2005trust,marcotte2001trust}. Most of the classical approaches deal with convex problems. Some strong assumptions such as continuous differentiability and lower semi-continuity are also quite common. Of those methods, single level reduction is the most widely used one when the lower level problem is convex. The single level reduction method is applied in this work. The second category applies evolutionary methods including genetic algorithms \cite{mathieu1994genetic}, particle swarm optimization \cite{li2006hierarchical}, differential evolution \cite{zhu2006hybrid}, and metamodeling-based methods \cite{wang2007review}. They normally require significant computational efforts and may not perform well. For a detailed review of different bilevel optimization methods, please refer to \cite{bilevel1, bilevel2}. 

There has been on-going research of bilevel optimization of in the power systems. In \cite{vahidinasab2009multiobjective,soleymani2008new,kozanidis2013mixed,zhang2011competitive,andrianesis2011mixed}, the bidding strategies of generation companys(Genco) are studied, in which the Genco's payoff is formulated as the upper level problem and the independent system operator's dispatch is formulated as the lower level problem. The planning and investment of different power system components using a bilevel optimization framework is investigated in \cite{verma2018information, zolfaghari2018bilevel, pandvzic2018investments, zhou2011designing, baringo2011wind}. \cite{yuan2011modeling, xiang2017coordinated, arroyo2009genetic, pinar2010optimization,arroyo2010bilevel} use bilevel optimization to study power system vulnerability issues which focus on minimizing the system loss under terrorist attacks on some transmission line or generator. The bilevel optimization work closely related to our work which centers on the transmission and distribution interaction appears in \cite{haghighat2012bilevel, li2007multiperiod, zhang2016trading}. In those work,  the distribution system optimal dispatch is the upper level problem, while the transmission system optimal dispatch is the lower level problem.  In \cite{haghighat2012bilevel}, the author uses bilevel optimization for the operational decision making of a distribution company(DS) in a competitive market with many other DSs.  A multi-period energy acquisition bilevel optimization model is proposed for a distribution company with distributed generation(DG) and interruptible load(IL) in \cite{li2007multiperiod}. The roles of DG and IL to alleviate congestion are also analyzed in the bilevel framework.  Trading strategies of DSs with distributed energy resources are examined in the day-ahead market and real time market in \cite{zhang2016trading}. For those work, there is a main problem with having the DS as the upper level problem. The upper level problem needs to know the lower level problem information such as GENCO information as well as other DSs's information, however it is unrealistic to assume the DS has that information. It is more intuitive and reasonable to have the transmission system operated by the independent system operation(ISO) as the upper level problem and the DS as the lower level problem since the ISO receives information from both the GENCO side and DS side and the ISO is at the top level in the power system hierarchy.

Int this work, the ISO or transmission system dispatch is formulated as the upper level problem, the DSs are formulated as the lower level problem. Under this structure, the upper level problem decides the energy and demand response prices of the DSs, the lower level problem responds by deciding the quantity of DR and energy import. The key contributions of this work are summarized below.

1. Present a bilevel optimization framework for the transmission and distribution system co-optimization.

2. Compare the bilevel optimization approach with the traditional co-optimization approach in different aspects. 

3. Reformulate the bilevel problem as a single level problem with KKT conditions and linearize the resulting nonlinear problem with the big-M method and strong duality theorem.

The structure of the paper is as follows; the problem under study is described in section~\ref{sec:pd}, the bilevel optimization model is discussed in section~\ref{sec:Model}.  Numerical results for are reported in Section~\ref{sec:Results}. Concluding remarks and future research directions follow in Section~\ref{sec:Conclude}. 