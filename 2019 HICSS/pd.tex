%!TEX root = HICSS51.tex
% conclusions

\subsection{Transmission System Problem}
The TS has a network of transmission lines and buses. Traditional and renewable generation units, loads and DS are connected to different buses in the network. The TS solves a day ahead market unit commitment problem, which is a co-optimization of the energy market and ancillary service market. For the energy market optimization, the TS tries to minimize the cost of meeting the system demand with its own generation or energy from DS. For the ancillary service market optimization, the TS minimizes the cost of providing enough reserve to account for the renewable forecast uncertainty. The reserve service could either come from the TS generator's reserve or the DS's DR. The TS objective function minimizes the energy and ancillary service cost at the same time.  

\subsection{Distribution System Problem}
The distribution system has dispatchable load and non-dispatchable load, and distributed generation in a radial network. In the day-ahead market, the DS solves an optimal dispatch problem with power flow. The dispatchable load is optimized at some point in between its upper and lower bound. The difference between the upper/lower bound and its set point could be used to provide upward/downward DR. The objective of the DS is to minimize the cost of meeting its demand either by its distributed generation or importing power from the TS and maximize the revenue of providing DR

\subsection{Co-operation Mode}
Under the bilevel co-optimization framework, the TS decides the price of DS energy import as well as the price for purchasing DS DR, the DS responds to those prices by exchanging a certain amount of energy with the TS and selling a certain amount of DR to the TS.

The co-optimization between the TS and DS is illustrated in Fig.~\ref{wees}
\begin{figure}
\centering
\includegraphics[scale=0.3]{flows.png}
\caption{Co-optimization between the TS and MG}
\label{wees}
\end{figure}

\subsection{Renewable Forecast Uncertainty Management}
In this work, we use a robust approach to manage the uncertainty in renewable forecast. Specifically, for a renewable generation forecast and a set of possible generation scenarios, we calculate the upward/downward forecast deviation by taking the difference of the hourly maximum/minimum generation scenario and the hourly forecast. The downward/upward TS generation reserve and upward/downward MG DR are used to account for the upward/downward renewable forecast deviation.

