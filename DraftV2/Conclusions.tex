%!TEX root = Journal.tex
% conclusions

This paper presents an analysis of a grid-connected microgrid with DR and distributed generation. 

\section*{Acknowledgements}
This material is based upon work supported by the US Department of Energy
under Award Number DE-OE0000843. Disclaimer: This report was prepared
as an account of work sponsored by an agency of the United States
Government. Neither the United States Government nor any agency thereof,
nor any of their employees, makes any warranty, express or implied, or
assumes any legal liability or responsibility for the accuracy, completeness,
or usefulness of any information, apparatus, product, or process disclosed,
or represents that its use would not infringe privately owned rights.
Reference herein to any specific commercial product, process, or service
by trade name, trademark, manufacturer, or otherwise does not necessarily
constitute or imply its endorsement, recommendation, or favoring by the
United States Government or any agency thereof. The views and opinions
of authors expressed herein do not necessarily state or reflect those of the
United States Government or any agency thereof.

%This paper compares three different pricing schemes for the energy exchange of the microgrids with the distribution system. Based on the Monte Carlo simulation results with real pricing, load and wind data, the RTP scheme outperforms the other two in all the cases in terms of the microgrid operation cost. The TOU scheme performs better than the FIX scheme in most cases, while for a few cases the TOU scheme fails to reflect the true pricing information and results in worse performance than the FIX scheme. Consequently, RTP sees the lowest average energy cost and FIX sees the highest. The difference in the performance is due to the ability of each pricing scheme to reflect the true market pricing condition. In this sense, RTP perfectly reflects the true market condition. TOU reflects the true market condition to some extent, while FIX could not reflect the market condition at all. The better the pricing scheme is at showing the true market condition, the more room the microgrid storage and self-generation have to arbitrage and reduce the operation cost. That is the reason for the largest cost saving with the storage and self-generation under RTP and not really any saving under FIX.\\\\
%There are currently not many microgrids and no RTP programs for microgrids in the US. As mentioned in the introduction, microgrids have a critical role in our future power network. This paper provides the foundation for the most efficient market design for the microgrid energy exchange for our future market. 

%This paper explored the effect of system flexibility, in terms of controllable loads and different risk levels, on the integration of significant wind resources within a chance-constrained unit commitment framework. Results have shown that the  availability of reserves from controllable loads is effective in enabling higher wind integration levels, but that increasing controllable loads to a higher percentage of total load does not provide further benefits. This is a result of the type of loads in this study -- the thermostatically controlled loads are realistically modeled with storage-like characteristics. The energy used from controllable loads for reserves must be returned to the system during the subsequent time period, thereby limiting the usefulness of additional reserves of this type. This finding leads to the conclusion that there is a need for responsive demand sources with different characteristics or costs, in addition to the controllable loads included in this study. 
%
%In addition, results show that the advantage of total reliability from robust solutions may not be synergistic with the societal goal of integrating significant uncertain renewable resources in the existing power system, due to its heavy burden on controllable generation. A more flexible solution, such as that provided by the chance-constrained formulation, can provide a middle ground between the deterministic-equivalent solution and the overly conservative robust solution. In addition to operational burden, the additional cost of robust solutions is significant over probabilistic solutions, even with high probability on the constraints. 
%
%While this simulation study has shown some interesting comparisons, and highlighted the importance of various classes of flexibility to wind integration, it also requires additional work to provide practical answers. Specifically, the controllable loads modeled here are a deterministic function of ambient temperature. It is widely acknowledged that DR is not a deterministic resource \cite{mathieu_uncertainty_2013}. Given the importance of this resource in the results presented here, it is clearly important to introduce uncertainty into the load side of this model, as was done in \cite{Vrakopoulou:2015gb, li:2015wh, zhang_data_2015}. Also other classes of controllable loads will be explored in future explorations within this framework. Finally, the results presented here are specific to the 57-bus system. While many of these findings are likely universal, the usefulness of a unit commitment model is limited if it is not scalable to networks of practical size. Therefore, decomposition methods will be explored to allow the application to large networks. 
 

